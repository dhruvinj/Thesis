\section{Model 2}

The results displayed in this section are for the uncertainty involved in the calculation of flamespeed depending  on two parameter i.e the activation energy for the fall off reaction in the ozone mechanism and activation energy for second reaction in the mechanism. The percentage of ozone is taken as 40, 46, 53, 75, and 100  percent acccording to the experimental data available to us from Streng\cite{Streng}.The results are displayed in five sections. In first section, the plots are shown for kernel density estimation of parameters $E_3$ and $E_2$. In second section, for constant surrogate size, the number of samples are changed from 1e5 to 1e7 and convergence is observed. In the third part of the results, convergence study is done for surrogates with different sizes. In fourth section, we ensure that samples of the parameter which we are drawing are fitting the flamespeed values of the experiment. In the fifth section mean and correlation plots are shown for the samples of both the parameters. The surrogates for individual concentrations are constructed using linear interpolation function. The initial guess for the map point is calculated using nelder mead optimization technique. After supplying initial guess over large domain it is found that the map point is the same no matter where we start our guess.


\subsection{ Statistics }

 In this section, we display results for sample size 1e7 and surrogate of 100X100 points. We  have plotted the kde.

\begin{figure}[H]
\centering
\subfloat[KDE \label{subfig-4:KDE}]{
        \includegraphics[scale=0.7]{model_2/kde}
            }
    \caption{Results for sample size 1e7}
\end{figure}

\subsection{Convergence Study- Number of samples }

 In this section, we see the convergence of the probability distribution as we increase the raw chain sample size. The plot is done for surrogate size of 100. In this analysis, raw chain size of $1e5$, $5e5$ , $1e6$, $5e6$ and $1e7$ is taken.

\begin{figure}[H]
\subfloat[Convergence For $E_2$ \label{subfig-1:E2_sample_conv}]{
        \includegraphics[scale=0.3]{model_2/sample_conv_E2}
            }
 \subfloat[Convergence For $E_3$ \label{subfig-1:E3_sample_conv}]{
        \includegraphics[scale=0.3]{model_2/sample_conv_E3}
            }
            \caption{Convergence with respect to sample}
\end{figure}


\subsection{Convergence Study- Surrogate }

In this section, we see the convergence of the surrogate. As we increase the number of points in the surrogate, results should be close for different surrogate sizes. The plot is done for surrogate size of 10X10, 20X20 30X30 , 100X100, 150X150 and 200X200. In this analysis, raw sample chain size is $1e6$.

\begin{figure}[H]
\subfloat[Surrogate Convergence For $E_2$ \label{subfig-1:E2_surrogate_conv}]{
        \includegraphics[scale=0.3]{model_2/surrogate_conv_E2}
            }
 \subfloat[ Surrogate Convergence For $E_3$ \label{subfig-1:E3_surrogate_conv}]{
        \includegraphics[scale=0.3]{model_2/surrogate_conv_E3}
            }
            \caption{Convergence with respect to number of surrogate points}
\end{figure}

\subsection{Flamespeed Data fit}

 It is necessary to ensure that the samples of the parameter which we are drawing are fitting the flamespeed values of the experiment. In this section, we calculate the flamespeed for all the parameters drawn using the surrogate generated before. We have taken $1e7$ sample size and calculated flamespeed for different concentrations of ozone.

 \begin{figure}[H]
   \subfloat[ Flame speed for 40 \% ozone \label{subfig-1:40}]{
        \includegraphics[scale=0.45]{model_2/flame_40.pdf}
       }
\subfloat[Flame speed for 46 \% ozone \label{subfig-2:46}]{
        \includegraphics[scale=0.45]{model_2/flame_46.pdf}
            }
\end{figure}


 \begin{figure}[H]
  \ContinuedFloat
   \subfloat[ Flame speed for 53 \% ozone \label{subfig-3:53}]{
        \includegraphics[scale=0.45]{model_2/flame_53.pdf}
       }
\subfloat[Flame speed for 75 \% ozone \label{subfig-4:75}]{
        \includegraphics[scale=0.45]{model_2/flame_75.pdf}
            }
\end{figure}


 \begin{figure}[H]
  \ContinuedFloat
   \subfloat[ Flame speed for 100 \% ozone \label{subfig-5:100}]{
        \includegraphics[scale=0.45]{model_2/flame_100.pdf}
       }
  \caption{Flamespeed Data fit}
\end{figure}



\subsection{Mean and Autocorrelation plots}

In this section, we show the mean of the samples and autocorrelation plots . The mean plot shows the initial instability due to burning period of MCMC and after that it remains constant. It shows us that we should be using atleast more than these number of samples for our analysis. The last figure shows the histogram and the various parameters of the distribution. For E3: Mean:  11.7, Std. Dev.: 7.81, Skewness:  -0.743 and Kurtosis:  1.29. For E2 stats:Mean:  18.9, Std. Dev.:  4.77, Skewness:  -1.38 and Kurtosis:  4.55.

 \begin{figure}[H]
   \subfloat[ Mean \label{subfig-1:mean}]{
        \includegraphics[scale=0.45]{model_2/M1_running_avg.pdf}
       }
\subfloat[Autocorrelation \label{subfig-2:auto}]{
        \includegraphics[scale=0.45]{model_2/M1_autocorr.pdf}
            }
            \caption{Mean and autocorrelation for sample size 1e6}
\end{figure}

 \begin{figure}[H]      
\subfloat[  $E_2$ distribution  \label{subfig-1:e2_distribution}]{
        \includegraphics[scale=0.43]{model_2/E2.pdf} 
            }  
 \subfloat[ $E_3$ distribution  \label{subfig-1:e3_distribution}]{
        \includegraphics[scale=0.43]{model_2/E3.pdf}
            }
            \caption{$E_3$ and $E_2$ distributions}
\end{figure}
