Low mach number  asymptotics are performed by Muller~\cite{Muller} that it is shown that a multiple-time scale,
single-space scale asymptotic analysis of the compressible
Navier-Stokes equations reveals that the zeroth-order global
thermodynamic pressure, the divergence of velocity, and the material
change of density are affected by heat-release rate and heat
conduction at low-Mach-numbers. His results show that the acoustic time change of the
heat-release rate as the dominant source of sound in low-Mach-number
flow. The asymptotic expansion of all flow variables show that the
viscous and buoyancy forces enter the computation of the second-order
incompressible pressure in low-Mach-number flow in a similar way as
they enter the pressure computation in incompressible flow, except
that the velocity-divergence constraint is non zero. He averaged the flow
equations over an acoustic wave period and the resulting averaged velocity tensor
described the net acoustic effect on the averaged flow field. Once
acoustics were removed from the equations altogether it lead to the
low-Mach-number equations.

\bigskip
 The incompressibility assumption makes the problem simpler
than if a full compressible flow is considered. Codina~\cite{Codina}
proposed that for ideal fluids, with isentropic condition, solutions
of the incompressible Navier Stokes equations can be found as the
limit of solutions of the compressible ones as the Mach number tends
to zero under certain assumptions on the initial data. In this paper, 
they showed that small Mach number limit
gives rise to a separation of the pressure into a constant-in-space
thermodynamic pressure and a mechanical pressure that has to be used
in the momentum equation. This leads to a removal of the acoustic
modes and the flow behaves as incompressible, in the sense that the
mechanical pressure is determined by the mass conservation equation
and not by the state equation. However, large variations of density
due to temperature variations are allowed. They showed that when the
Mach number is small the hyperbolic wave equation for the pressure
becomes an elliptic equation for the first order pressure, thus
showing the implicit (“incompressible” or “mechanical”) character of
this pressure component.


Measurements of laminar flame speeds is an important property of a
reacting mixture in the combustion regime. The laminar flame velocity is defined as the
velocity of the unburned gases through the combustion wave in the
direction normal to the wave surface. One of the first techniques to measure the laminar flame
speed is the Bunsen burner technique. Approximating the conical
surface area of the innermost cone, the cross sectional area of the
burner tube, and the average flow velocity in the tube, the laminar flame
speed could be calculated. However, the accuracy obtained by this
method is of $\pm $ 20 \% according to Kuo~\cite{Kuo}. The next method
of calculating the flame speed is burning the mixture of oxygen
and fuel in a tube. This method is known as constant volume spherical
bomb method~\cite{Kuo}. The time variation of instantaneous mass of
unburned gas is proportional to the change of pressure and change of
spherical flame radius with respect to time. The flame speed is
calculated as a difference in the two time derivatives. The
disadvantage of this method is that error in the time derivative can be
magnified in calculations of laminar flame speed.

 In the soap bubble method, a homogeneous combustible mixture is
allowed to blow a soap bubble around a pair of spark electrodes. At
time zero, the gas mixture contained in the spherical soap bubble is
ignited by a spark. From the average spatial velocity of the spherical
flame front, the initial radius of the soap bubble and the final
radius of the sphere of the burned gases, the laminar flamespeed is
calculated. The initial and final sizes of the bubble must be known
very accurately. The method is not suitable for studying the flame
spread of dry mixtures as there is heat loss due to electrodes.

 Lewis and Von Elbe~\cite{lewis} used rectangular
burners. They devised a particle track method in which small magnesium
oxide particles in the gas stream were illuminated intermittently from the
sides. A photograph of the particle shows its direction and the
velocity of the particle can be deduced from consecutive
photographs. The disadvantage of this method is that the added
particles bring about the catalytic change at the surface, modifying the combustion
processes and hence changing the laminar flame speed.

 Powling~\cite{powling} invented the flat flame burner
method. It is most accurate method due to a very simple flame
front. The area of shadow, schlieren, and visible fronts are all the
same. The gaseous mixture is usually ignited at high flow rate and
adjusted until the flame is flat. This method is applicable only to
the mixtures having low burning velocities. Spalding and
Botha~\cite{spalding} extended the flat flame burner method to higher flame
speeds by cooling the porous disk which brings the flame front closer to
the disk. The values of flame speed are plotted at different cooling
rates. The curve is then extrapolated to zero cooling rate to obtain
adiabatic flame speed.

 Over the past 20 years, there have been many technological
advancements in the field of laser based optical techniques, computer
coupled data acquisition systems and digital image capture and
analysis. Eckbreth~\cite{eckbreth} has used laser spectroscopy to
determine the laminar flame speed. Other velocity measurement
techniques include hot wire anemometer, laser doppler anemometry (LDA),
and particle image velocimetry with a pulsed thin laser sheet.


 The evaluation of laminar flame speed is important for 
two reasons. First is that the laminar flame speed is the key
input parameter to various models of premixed turbulent
combustion. Secondly, data on laminar flames are widely used to calibrate
detailed and semi detailed chemical mechanisms of combustion of
various fuels. Thus to design an engine, to model turbulent combustion,
and for validation of chemical kinetics, it is necessary to have an
accurate knowledge of the laminar flame speeds. Over the years,
reported values of laminar flame speeds have been characterized by
substantial scatter~\cite{Andrew}. This scatter is observed because it
is not possible to generate 1D planar, adiabatic, steady laminar
flames for which flame speed is defined. Most of the flames are
stabilized by heat loss and curvature. The counterflow technique
introduced by Law and co-workers~\cite{law} gave laminar flame speeds
that are used extensively for the validation of chemical kinetics and
modeling of turbulent combustion. In that work, it is necessary to
take measurements at low strain rates so that linear extrapolation to
zero strain rate produces accurate values of laminar flame
speed. Dixon ~\cite{dixon} and Tien~\cite{tien} showed that
extrapolations can lead to over prediction if true value of laminar
flame speed.

Mosbach et al ~\cite{mosbac} applied a Bayesian parameter estimation technique to a chemical kinetic mechanism for n-propylbenzene oxidation in a shock tube. A Bayesian methodology used to study the influence of uncertainty in experimental measurements. They showed that  due to the challenges associated with producing the surrogates of sufficient fidelity over large range in parameter space, optimization is done before applying Bayesian method. Response uncertainties are significantly underestimated. Second-order response surfaces are sufficiently accurate but not always. Assuming normal distribution for propagated errors is largely adequate but not in all cases. Becker et al ~\cite{Becker2005} used local mesh refinement controlled by posterior error estimation with respect to errors in parameters for parameter identification in combustion problems modeled by partial differential equations. They applied this algorithm to two combustion problems namely, identification of Arrhenius parameters and diffusion coefficient for ozone flame calibration. Local mesh refinement was used for finding efficient discretizations for parameter identifications. When the prescribed accuracy is achieved, an optimization loop is applied. Then the unknown parameters were determined by intrinsically coupled mesh refinement algorithm. Reagan et al ~\cite{Reagan} discussed two techniques for uncertainty quantification in chemistry computations. One of the techniques was based on sensitivity analysis and error propagation while the other was based on stochastic analysis using polynomial chaos (PC) techniques. They reported comparison is in terms of the predictions of uncertainty in species concentration and sensitivity of selected species concentration to given parameters. The PC expansion construction allowed extractions from contribution of each uncertain parameter to the uncertainty in the sensitivity coefficients. Significant under-prediction of uncertainty in the error propagation technique was observed as compared to the PC construction. 


