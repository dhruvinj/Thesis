\section{Literature review}

It is shown by Muller~\cite{Muller} that a multiple-time scale, single-space scale asymptotic analysis of the compressible Navier-Stokes equations reveals that the zeroth-order global thermodynamic pressure, the divergence of velocity; and the material change of density are affected by heat-release rate and heat conduction at low-Mach-numbers.

\bigskip 
\noindent His result show that the acoustic time change of the heat-release rate as the dominant source of sound in low-Mach-number flow. The asymptotic expansion of all flow variables show that the viscous and buoyancy forces enter the computation of the second-order incompressible pressure in low-Mach-number flow in a similar way as they enter the pressure computation in incompressible flow, except that the
velocity-divergence constraint is non zero. He averaged flow equation over an acoustic wave period, the averaged velocity tensor described the net acoustic effect on the averaged flow field. Once acoustics were removed from the equations altogether it lead to the low-Mach-number equations.

\bigskip
\noindent The incompressibility assumption makes the problem simpler than if a full compressible flow is considered. Codine\cite{Codina} proposed that for ideal fluids, with isentropic condition, solutions of the incompressible Navier Stokes equations can be found as the limit of solutions of the compressible ones as the Mach number tends to zero under certain assumptions on the initial data.

\noindent In this paper, they showed that small Mach number limit gives rise to a separation of the pressure into a constant-in-space thermodynamic pressure and a mechanical pressure that has to be used in the momentum equation. This leads to a removal of the acoustic modes and the flow behaves as incompressible, in the sense that the mechanical pressure is determined by the mass conservation
equation and not by the state equation. However, large variations of density
due to temperature variations are allowed. They showed that when the Mach number is small the hyperbolic wave equation for the pressure becomes an elliptic
equation for the first order pressure p(2), thus showing the implicit (“incompressible” or “mechanical”) character of this pressure component.


\noindent The Laminar flame speed is the important property of a reacting mixture. The laminar flame velocity is defined as the velocity of the unburned gases throught the combustion wave in the direction normal to the wave surface. 

\noindent The one of the first techniques to measure the laminar flame speeds is the bunsen burner technique. Approximating the conical surface area of the innermost cone, the cross sectional area of the burner tube, and the average flow velocty in the tube, Laminar flame speed could be calculated. However the accuracy obtained by this method is of $\pm $ 20 \% according to Kuo\cite{Kuo}. The next method of calculating the flame speed is the burning the mixture of oxygen and fuel in a tube. This method is known as constant volume spherical bomb method\cite{Kuo}. The time variation of instantaneous mass of unburned gas is propotional to the change of pressure and change of spherical flame radius with respect to time. The flame speed is calculated as a difference in the two time derivatives. The disadvantage of this method is that error in the time deivative can be magnified in calculations of laminar flame speed. 

\noindent In soap bubble method, a homogeneous combustible mixture is allowed to blow a soap bubble around a pair of spark electrodes. At time zero, the gas mixture contained in the spherical soap bubble is ignited by spark. From the average spatial velocity of the spherical flame front, the initial radius of the soap bubble and the final radius of the sphere of the burned gases, the laminar flamespeed is calculated. The initial and final sizes of the bubble must be known very accurately. The method is not suitable for studying the flame spread of dry mixtures. There is heat loss due to electrodes. 

\noindent Lewis and von Elbe\cite{lewis} used rectangular burners. They devised a particle track method in which small magnesium oxide particles in gas stream were illuminated interittently from the sides. A photograph of the particle shows its direction and the velocity of the particle can be deduced from consecutive photographs. The disdvantage of this method is that the added particles ay bring about the catalytic change in the combustion processes and hence change the laminar flame speed. 

\noindent Powling\cite{powling} invented the flat flame burner method. It is most accurate method due to a very simple flame front. The area of shadow, schlieren, and visible fronts are all the same. The gaseous mixture is usually ignited at high flow rate and adjusted untill the flame is flat. This method is applicable only to the mixtures having low burning velocities. Spalding and Botha\cite{splading} extended flat flame burner method to higher flame speeds by cooling the porous disk which brings flame front closer to the disk. The values of flame speed are plotted at different cooling rates. The curve is then extrapolated to zero cooling rate to obtain adiabatic flame speed. 

\noindent Over past 20 years, there has been a lot of technological advancements in the field of laser based optical techniques, computer coupled data acquisition systems and digital image capture and analysis. Eckbreth\cite{eckbreth} has used laser spectroscopy to determine the lamianr flame speed. Other velocity measurement techniques include Hot wire anemometer, Laser Doppler anemometry (LDA) and particle image velocimetry with pulsed thin laser sheet. 


\noindent The evaluation of laminar flame speed is important due to two important reason. First is that, laminar flame speed is the key input parameter of various models of premixed turbulent combustion. Secondly, data on laminar flames are widely used to access detailed and semi detailed chemical mechanisms of combustion of various fuels. Thus to design engine, to model turbulent combustion and for validation of chemical kinetics it is necessray to have an accurate knowledge of the laminar flame speeds. Over the years, reported values of laminar flame speeds have been characterised by substantial scatter\cite{Andrews}. This scatter is observed because it is not poosible to generate 1D planar, adiabatic, steady laminar flames for which flame speed is defined. Most of the flames are stabilized by heat loss and curvature. The counterflow technique introduced by  Law and co-workers\cite{law} gave laminar flame speeds that are used extensively for the validation of chemical kinetics and modeling of turbulent combustion. In this work, it is necessary to take measurements at low strain rates so that linear extrapolation to zero strain rate produces accurate values of laminar flame speed. Dixon \cite{dixon} and Tien\cite{tien} showed that extrapolations can lead to over prediction if true value of laminar flame speed.  

