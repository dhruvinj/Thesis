

%%%%Introduction
In this work, we identify the probabilistic representation of the uncertain parameters in the ozone combustion model. We  apply the Bayesian methodology to study the influence of
uncertainties in the experimental measurements of the flame speed on chemical kinetic rate of chemical reactions involved. These kinetic rates contain uncertain parameters like activation energy and pre exponential factors. Given a set of experimental data on flamespeeds for different concentrations of ozone, this framework, allows for the
calibration of model parameters, determination of uncertainty in those parameters, and propagation
of that uncertainty into simulations.

\bigskip
%%%%%Methodology

We have described the mathematical models used for combustion for both multi dimensional and one dimensional reacting-flow . We will use the mathematical model developed for 1D free flame~\cite{Kuo}. Numerical approximation developed by Smooke~\cite{Smooke} are used. We use Cantera~\cite{Cantera} to solve for forward model. Due to sampling load and the complex combustion model, We need not solve for the full model always. In this work, we show the use of surrogate models. We use linear surrogate models to calculate the flame speed values at different points in the domain. We use three different models for uncertainty quantification in which we gradually increase the dimensions of our problem. We begin with doing uncertainty analysis for the most sensitive parameter and then go the uncertainty quantification for two parameter case. In the last model, we do uncertainty analysis for two most sensitive parameter of our study.  


\bigskip


Delayed accept reject Markov chain Monte Carlo method 
algorithms are employed to sample from the posterior probability densities with the help of QUESO~\cite{QUESO}, making use of
linear surrogate fitted to the model responses. We analyze the  probability density functions of the solution. We
also examine the role and significance of dependence among the uncertain parameters. After calibration the resulting uncertainty in the parameters is propagated forward into
the simulation of laminar flame speeds and these values are compared as the experimental data. They are analyzed whether they have a good fit to the data or not. 


\bigskip


%%%Results%%%%%%

We conclude that this methodology provides a useful tool for the analysis of distributions of model parameters and
responses, in particular their uncertainties and correlations. We conclude that using surrogate models instead of forward model helps to reduce the computational load and gives a reasonable results. We also show that by using these surrogate models, we have the necessary convergence for both samples and surrogates. We analyze the effect of increase of dimension of the problem on the number of samples required for the good statistical estimate of the parameter. 
