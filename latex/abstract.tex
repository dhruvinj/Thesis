

%%%%Introduction
In this work, we identify the probabilistic representation of uncertain parameters in an ozone combustion model. We  apply a Bayesian methodology to study the influence of
uncertainties in the experimental measurements of laminar flame speed on the calibration of chemical kinetic rates of chemical reactions. These kinetic rates contain uncertain parameters including the activation energy and the pre-exponential factor. Given a set of experimentally determined flamespeeds for different concentrations of ozone, this framework allows for the
calibration of model parameters, determination of uncertainty in those parameters, and propagation
of that uncertainty into simulations.

\bigskip
%%%%%Methodology

We have described the mathematical models used for combustion including multi-dimensional and one-dimensional reacting-flow. We present a mathematical model developed for a one-dimensional free flame. We use Cantera to solve the forward model. Due to complex combustion model and the large number of required samples, we need not use the full model for every sample. In this work, we show the use of surrogate models whereby we use linear interpolation to calculate the flame speed values at different points in parameter space. We use three different statistical models in which we gradually increase the dimensions of our problem. We begin with the most sensitive parameter and then proceed to the two parameter case. In the last model, we perform inference for two most sensitive parameters of our study.  


\bigskip


Markov Chain Monte Carlo (MCMC) 
algorithms are employed to sample from the posterior probability distributions. We use the C++ statistical library QUESO, to generate 
linear  interpolation surrogates and subsequent MCMC sampling. We analyze the probability density functions of the solution. We
also examine the role and significance of dependence among the uncertain parameters. After calibration, the resulting uncertainty in the parameters is propagated forward into
the simulation of laminar flame speeds and these values are compared with the experimental data to ensure a reasonable calibration.


\bigskip


%%%Results%%%%%%

We conclude that this methodology provides a useful tool for the analysis of distributions of model parameters and
responses, in particular their uncertainties and correlations. We conclude that using surrogate models instead of forward model helps to reduce the computational load and gives reasonable results. We demonstrate adequate convergence both in the number of points used in the surrogate model and in the number of MCMC samples. We analyze the effect of increase of dimension of the problem on the number of samples required for the good statistical estimation of the parameters. 
