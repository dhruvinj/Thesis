\documentclass[10pt]{ubthesis}
\usepackage{graphicx}
\usepackage{subfig}
\usepackage{float}
\title{TOWARDS UNCERTAINTY QUANTIFICATION OF PREMIXED LAMINAR FLAMES USING BAYESIAN STATISTICS}

\author{Dhruvin Jasmin Naik}

\conferral{\today}

\dept{Mechanical and Aerospace Department}

\degree{ms}

\begin{document}
\begin{titlepage}
\maketitle
\end{titlepage}

\begin{ubfrontmatter}
\makecopyright
\cleardoublepage
\begin{acknowledgements}
These are my acknowledgements
\end{acknowledgements}
\tableofcontents

\cleardoublepage
\listoffigures
\cleardoublepage
\listoftables
\cleardoublepage
\begin{abstract}
This is my abstract
\end{abstract}
\end{ubfrontmatter}
\chapter{Introduction}

It is important to understand deeply the dynamics of reacting flows as it is inherent in number of areas of science and engineering like combustion, surface chemistry etc . To understand these systems in detail, large simulations are necessary. It also requires simultaneous numerical resolution of chemical reactions, diffusive
transport and fluid mechanics. The combination of three factor make simulations some of the most demanding in the area of combustion. In the deflagration (combustion wave that propagates through a gas or across the surface of an explosive at subsonic speeds, driven by the transfer of heat) regime where the burning speed (also known as laminar flame speed) and associated velocity scales are much smaller than the speed of sound in the fluid, the problem is acute.

\bigskip
\noindent For low speed combustion flows, Low Mach number asysmptotics of the flow equations exploit the inherent separation of scales in such systems by analytically eliminating acoustic wave propagation entirely from the dynamics, while preserving the important compressibility effects arising from reactions and transport.
\section{Literature Review}
It is shown by Muller that a multiple-time scale, single-space scale asymptotic analysis of the compressible Navier-Stokes equations reveals that the zeroth-order global thermodynamic pressure, the divergence of velocity and the material change of density are affected by heat-release rate and heat conduction at low-Mach-numbers.

\bigskip 
\noindent His result show that the acoustic time change of the heat-release rate as the dominant source of sound in low-Mach-number flow. The asymptotic expansion of all flow variables show that the viscous and buoyancy forces enter the computation of the second-order incompressible pressure in low-Mach-number flow in a similar way as they enter the pressure computation in incompressible flow, except that the
velocity-divergence constraint is non zero. He averaged flow equation over an acoustic wave period, the averaged velocity tensor described the net acoustic effect on the averaged flow field. Once acoustics were emoved from the equations altogether it lead to the low-Mach-number equations.

\bigskip
\noindent As many flows of interest can be considered as incompressible. The incompressibility assumption makes the problem simpler than if a full compressible flow is considered. Codine  proposed that for ideal fluids, with isentropic condition, solutions of the incompressible Navier Stokes equations can be found as the limit of solutions of the compressible ones as the Mach number tends to zero under certain assumptions on the initial data.

\noindent In her paper she showed that small Mach number limit gives rise to a separation of the pressure into a constant-in-space thermodynamic pressure and a mechanical pressure that has to be used in the momentum equation. This leads to a removal of the acoustic modes and the flow behaves as incompressible, in the sense that the mechanical pressure is determined by the mass conservation
equation and not by the state equation. However, large variations of density
due to temperature variations are allowed. She showed that when the Mach number is small the hyperbolic wave equation for the pressure becomes an elliptic
equation for the first order pressure p(2), thus showing the implicit (“incompressible” or “mechanical”) character of this pressure component

\bigskip
\noindent A.G Streng and A.V. Grosse  studied the ozone oxygen flame experimentally. The stability of ozone and the rates of decomposition or explosion were investigated by Armour research foundation. Ozone was burned to oxygen from a simple burner tip in the range from 17 percent to 100 percent initial concentration of ozone in the mixture. The flame temperatures were calculated from enthalpy data and dissociation constants of oxygen using kelley's tables. The concentration of ozone was kept constant with error of   0.2 percent. Two methods were used to determine burning velocity i.e open tube method and the burning tip method. We will be using experimental burning velocity of the burning tip method. The burner tip experiments were carried out in standard apparatus, using pyrex glass aluminium tips with an inner diameter of 3 to 0.65mm. The flames were readily observed by the standard schlieren method at all concentrations above 30 mole percent. The measurements were all carried out in the laminar flow region and the reynolds number of the flow was below 2000. The initial conditions are 300K temperature and 1.0 atmosphere pressure. The results of burning velocities of ozone flames were compared with theoretical burning velocities of Dr. Von Karman and his associates. They were found to be in close agreements. 

\section{Governing Equations}
The flow of a compressible fluid is described in terms of the velocity $(u^*)$, pressure $(p^*)$, density $(\rho^*)$, and temperature $(T^*)$ fields. These fields are solutions of the compressible Navier-Stokes equations that describe the dynamics of the system
and that are statements of conservation of mass, momentum, and energy and a state equation relating the thermodynamic variables.  

The system of equations that needs to be solved reads
\begin{eqnarray}
\frac{D \rho^*}{D t^*}\hspace{1mm} \hspace{1mm} &=& - \rho^* \hspace{1mm}\nabla \cdot u^* \\
\rho^* \frac{D u^*}{D t} \hspace{1mm} &=& -\hspace{1mm}\nabla p^*\hspace{1mm}+\hspace{1mm}\nabla \cdot \tau^* + \rho^* g^* \\
\rho^* C_p^* \frac{D T^*}{D t^*} \hspace{1mm}&=&\hspace{1mm} \tau^* \nabla \cdot u^* \hspace{1mm}- \hspace{1mm}\nabla \cdot q^* + \hspace{1mm} \beta^* T^* \frac{D p^*}{D t^*}\hspace{1mm}\\
\frac{D \rho^*}{D t^*} &=& \hspace{1mm} - \beta^* T^* \frac{D \rho^*}{D t^*} + \alpha^* \rho^* \hspace{1mm}\frac{D p^*}{D t^*}
\end{eqnarray}

Where $u^*$ is the velocity vector, $\rho^*$ is the density, $\beta^*$ is the thermal expansion coefficient,$p^*$ is the pressure , $\tau^*$ is the viscous stress term, $g^*$ is the gravitational vector, $c_p^*$ is the specific heat, $T^*$ is the temperature, $q^*$ is the heat flux vector, $\mu^*$ is the dynamic viscocity. $k^*$ is the thermal conductivity.

\bigskip

For a Newtonian fluid, $$\tau^* = \mu^*(\nabla u^* +(\nabla u^*)^T)) - \frac{2}{3}\mu^*\nabla \cdot u^*I$$

By Fourier law, 
$$q^* =  k^* \nabla T^*$$

\bigskip

 The following are the quantities in equation of state: $\beta^* = -\frac{1}{\rho^*}\frac{\partial \rho^*}{\partial T^*}$ and $\alpha^* = \frac{1}{\rho^*}\frac{\partial \rho^*}{\partial p^*}$.

\section{Nondimensionalization}

We nondimensionalize the equations by using reference quantities denoted by the subscript $\infty$,
e.g. farfield or stagnation conditions, and a typical length scale $L^*$ of the considered
flow. The thermodynamic reference quantities are assumed to be related by the equation of
state for a perfect gas. We have used perfect gas law because we are dealing with gases in a combustion environment. We define the nondimensional quantities by\\

\bigskip

$\rho = \frac{\rho^*}{\rho_\infty} $, \hspace{2mm} $p = \frac{p^*}{p_\infty} $, \hspace{2mm} $u = \frac{u^*}{u_\infty} $, \hspace{2mm} $T = \frac{T^*}{T_\infty} $, \hspace{2mm} $\mu = \frac{\mu^*}{\mu_\infty} $,\hspace{2mm} $k = \frac{k^*}{k_\infty} $,\hspace{2mm}$x = \frac{x^*}{L^*} $, \hspace{2mm} $t = \frac{t^*}{L^*/u^*_{\infty}} $, \hspace{2mm} $\beta^* = \frac{\beta}{\beta_{\infty}} $, 
\bigskip

$C_p^* = \frac{C_p}{{C_p}^*_{\infty}}$.

Using the relations above; we may write the nondimensional
Navier-Stokes equations and other equations of interest as follows:
\begin{eqnarray}
\frac{D \rho}{D t}\hspace{1mm} &=& -\hspace{1mm}\rho \hspace{1mm}\nabla \cdot u \\
\rho \frac{D u}{D t} \hspace{1mm} &=& -\frac{1}{M^2}\hspace{1mm}\nabla p\hspace{1mm}+\hspace{1mm}\frac{1}{Re}\nabla \cdot \tau + \frac{1}{Fr^2}\rho g \\
\rho C_p \frac{D T}{D t} \hspace{1mm}&=&\hspace{1mm} \frac{M^2}{Re \lambda}\hspace{1mm}\tau \nabla \cdot u \hspace{1mm}-\hspace{1mm}\frac{1}{Re Pr} \hspace{1mm}\nabla \cdot (k \nabla T) + \hspace{1mm} \frac{\beta T}{\lambda} \frac{D p}{D t}\hspace{1mm} \\
\frac{D \rho}{D t} &=& \hspace{1mm} - \beta T \frac{D \rho}{D t} + \alpha \rho \hspace{1mm}\frac{D p}{D t}
\end{eqnarray}
where the following nondimensional quantities are

$M= \frac{u_\infty}{a_\infty}$, \hspace{2mm} $Re= \frac{u_\infty \rho_\infty L}{\mu_\infty}$, \hspace{2mm}$Pr= \frac{{C_p}_\infty \mu_\infty L}{k_\infty}$, \hspace{2mm} $Fr= \sqrt{\frac{u_\infty^2} {g_\infty L}}$.

 $a_{\infty}$ is the reference speed of sound, $M$ is the mach number, $Re$ is the reynolds number, $Pr$ is the prandtl number, $Fr$ is the froude number, and $\lambda$ is defined as $\frac{{C_p}_\infty T_\infty }{a^2_\infty}$. 

 From ideal gas law, 

$$\beta^*= -\frac{1}{\rho^*}\frac{\partial \rho^*}{\partial T^*}$$ We non dimensional this term as follows 

$$\beta^*= -\frac{1}{\rho_\infty \rho}\frac{\rho_\infty}{T_\infty}\frac{\partial \rho}{\partial T}$$

$$\beta^*= \frac{1}{T_\infty } \beta$$ Where $\beta_\infty = \frac{1}{T_\infty}$
\bigskip

 From ideal gas law, 

$${a^*}^2= \frac{\partial p^*}{\partial \rho^*}$$ We non dimensional this term as follows 

$${a^*}^2= \frac{p_\infty}{\rho_\infty}\frac{\partial p}{\partial \rho}$$

$${a^*}^2= \frac{p_\infty}{\rho_\infty}a^2$$ Where $a_\infty = \sqrt{\frac{p_\infty}{\rho_\infty}}$.

\section{Asymptotic Analysis}
We do asymptotic analysis with respect to Mach number.
We use the following equation in terms of Mach number:

$$\xi(x,t,M)=\xi^0(x,t)+M \xi^1(x,t)+M^2 \xi^2(x,t)+O(M^3)$$ \\

where $\xi$ can be $u,p,\rho$. We use non dimensional form of Navier-Stokes equation do the asymptotic equation. 
We show the following example of asymptotic analysis. For mass conservation, 

$$\frac{D}{D t}(\rho^0+M \rho^1+M^2 \rho^2+O(M^3))\hspace{1mm}+\hspace{1mm}(\rho^0+M \rho^1+M^2 \rho^2+O(M^3)) \hspace{1mm}\nabla \cdot (u^0+M u^1+M^2 u^2+O(M^3)) = 0 $$

Collecting equal order terms together we get

$$\left(\frac{D \rho^0}{D t}\hspace{1mm}+\hspace{1mm}\rho^0 \hspace{1mm}\nabla \cdot u^0\right) + M \left(\frac{\partial \rho^1 }{\partial t}+ +\hspace{1mm}\rho^1 \hspace{1mm}\nabla \cdot u^0++\hspace{1mm}\rho^0 \hspace{1mm}\nabla \cdot u^1\right) + O(M^2) = 0 $$ Therefore zeroth order terms can be written.
\begin{eqnarray}
\frac{D \rho^0}{D t}\hspace{1mm}+\hspace{1mm}\rho^0 \hspace{1mm}\nabla \cdot u^0 &=& 0
\end{eqnarray}

 The asymptotic relation for momentum are obtained similarly.  The zero, first, and second order momentum equation are written as follows 
\begin{eqnarray}
M^{-2}\nabla p^{(0)} &=& 0\\
M^{-1}\nabla p^{(1)} &=& 0\\
\rho^0 \frac{D u^0}{D t} \hspace{1mm} &=& -\hspace{1mm}\nabla p^2\hspace{1mm}+\hspace{1mm}\nabla \cdot \tau^0 + \rho^0 g
\end{eqnarray}

 The asymptotic relation for the energy equation is
\begin{eqnarray}
\rho^0 C_p^0\frac{D T^0}{D t} \hspace{1mm}&=&\hspace{1mm}-\hspace{1mm}\frac{1}{Re Pr} \hspace{1mm}\nabla.(k^0 \nabla T^0) + \hspace{1mm} \frac{\beta^0 T^0}{\lambda} \frac{d p^0}{d t}\hspace{1mm}
\end{eqnarray}
We have seen from momentum asymptotic analysis that $p^0=p^0(t)$ and therefore $$\frac{D p^0}{D t}= \frac{d p^0}{d t}$$. 
 
 The nondimensional equation of state is same as that of the dimensional form. Now, the same asymptotic expansion of the state equation is done in order to get zeroth order state equation: 
\begin{eqnarray}
\frac{D \rho^0}{D t} &=& \hspace{1mm} - \beta^0 T^0 \frac{D \rho^0}{D t} + \alpha^0 \rho^0 \hspace{1mm}\frac{D p^0}{D t}
\end{eqnarray}

 From the mass conservation equation and since $p^0=p^0(t)$, 
\begin{eqnarray}
- \nabla \cdot u^0 &=& \hspace{1mm} - \beta^0  \frac{D T^0}{D t} + \alpha^0  \hspace{1mm}\frac{d p^0}{d t}
\end{eqnarray}

 From the energy conservation equation,
\begin{eqnarray}
- \rho^0 C_p^0 \nabla \cdot u^0 &=& \hspace{1mm}  \frac{1}{Re Pr}\nabla \cdot q^0 + \left(\frac{C_p^0 \lambda -1 }{\lambda}  \right)\hspace{1mm}\frac{d p^0}{d t}
\end{eqnarray}

The final equation obtained by applying various thermodynamic properties is
\begin{eqnarray}
-\nabla \cdot u &=& \frac{1}{p^0} \frac{d p^0}{d t} - \frac{1}{T}\frac{D T}{D t} \hspace{1mm}\\
\rho \frac{D u}{D t} \hspace{1mm} &=& -\frac{1}{M^2}\hspace{1mm}\nabla p\hspace{1mm}+\hspace{1mm}\frac{1}{Re}\nabla \cdot \tau + \frac{1}{Fr^2}\rho g \\
\rho C_p \frac{D T}{D t} \hspace{1mm}&=&\hspace{1mm} -  \frac{1}{Re Pr}\nabla\cdot (k \nabla T) + \hspace{1mm} \frac{1}{\gamma -1} \frac{d p^0}{d t}\hspace{1mm}\\
\end{eqnarray}

\chapter{Combustion Model}
\section{Premixed ozone oxygen laminar flame model}
We shall use Heimerl and coffee's contemporary method for modelling combustion flame problem. A one dimensional, premixed, laminar, steady state ozone oxygen flame was considered in their theortical model. The reason for choosing this model was due to its simplicity. The chemical reactions involved only three species 


			$$O_3 + M \rightleftharpoons O + O_2 + M $$
			$$ O + O_3 \rightleftharpoons A_2  2O_2$$ 
			$$A + AB \rightleftharpoons A_2 + B$$
\noindent Where M represents the third body which could be either $O$, $O_2$ or $O_3$. 

\noindent The species conservation equation is given as 
\begin{eqnarray}
	\frac{\partial \rho_i}{\partial t} + \nabla . (\rho_i v) &= \nabla . (\rho D_{ij} \nabla Y_i) + w_i
\end{eqnarray}
\noindent Where, the rate of production for species $i$ is given by the following 
	
Where $$w_i = Mw_i \sum_{k=1}^{6}(v_{k,i}^{''} - v_{k,i}^{'}) B_k T^{\alpha_k} e^{\frac{-E_a}{R_u T}} \prod_{j=1}^3 \left(\frac{X_j p}{R T} \right)^{v'_{j,k}}$$

\noindent The above equation considers all 6 reactions for all the three species. 

\noindent From atomic species conservation 
	$$\sum_{i=1}^{N} v_i{'}M_i \rightleftharpoons \sum_{i=1}^{N} v_i{''}M_i $$
\noindent Where $E_a$ is the activation energy, $R_u$ is the universal gas constant, $X_j$ is the molar concentration of the reactant j, $Mw_i$ is the molecular weight of the reactant. $v'_{j,k}$ is the moles of reactant j in the reaction k. $v''_{j,k}$ is the number of moles of product. 

\bigskip

\noindent The overall continuity equation, species conservation equation and energy conservation equation are solved to calculate the flame paraneters. The energy equation considers that the pressure is constant throughout the reaction zone. The viscous dissipation is negligible and there is no body force work. 

\noindent The Diffusion equation is given 
\begin{eqnarray}
\frac{\partial X_k}{\partial x} &= \sum_{j=1}^{3} \frac{X_k X_j}{D_{jk}} (V_j - V_k)
\end{eqnarray}

\noindent Where $D_{jk}$ is the fick's diffusion coefficient. $V$ is the diffusion velocities. 

\noindent The boundaries are defined by specifying the concentration of $O_3$, $O_2$ at the inlet. 

\section{The experimental data on laminar flame speed}

To compare our results we have used the experimental data given by the A.G.  streng and A.V. Grosse, They have done experiments with ozone flame in tube and ozone flame on the tip of the burner. We will be using the results of the later. They have shown that laminar flame speed or burning velocity varies with the initial concentration of the ozone. The table given below shows the burning velocity with respect to initial concentration of the ozone. We concentrate on two speciifc cases where initial concentration of ozone is 53 percent and 100 percent. The laminar flame speed for 53 percent is measured in the burner with inner diameter 1.3mm  and rate of 7.7 cc/sec. The laminar flame speed for 100 percent ozone is taken on .66 inner diameter tip 0.66 mm and the flow rate of 8.23 cc/sec. 
The measured laminar flame speed is given below. Laminar flame measurements are carried out at 300K and 1 atmosphere pressure. 

\begin{table}[h]
\caption {Experimental laminar flame speed given by A.G. Streng and A.V. Grosse} \label{tab:title}
\begin{center}

\begin{tabular}{|c|c|}
\hline
 \textbf{ Initial Concentration of $O_3$ ($\pm 0.2 \% )$}  &  \textbf{ Laminar Flame Speed (cm/s)} \\ \hline
 17& 9.2 \\  \hline
 20& 18.2 \\ \hline
 28& 52.2 \\ \hline
 40& 125 \\  \hline
 46&  166 \\ \hline
 53&  210 \\ \hline
 75&  331 \\ \hline
100&  475 \\ \hline
\end{tabular}
\end{center}
\end{table}

\section{Geometry with boundary conditions}

\begin{figure}[h!]
  
  \centering
   \includegraphics[scale=0.5]{geometricbcs}
   \caption{Geometry}
\end{figure}
\begin{itemize}
\item $AB$ is the inlet
\item $BC$ is the thickness of the burner tube
\item $CD$ is the outside of the burner
\item $DE,EF$ is the domain of interest
\item $AE$ is the Axisymmetric line
\end{itemize}

\bigskip
\noindent The following assumption are made
\begin{itemize}
\item At the inlet, the incoming velcity has parabolic profile. i.e poissuelle flow. 
\item The surface chemistry at the wall of the burner is neglected. 
\item At the outside of the burner, the incoming air has couette flow profile. 
\item $DE,EF$ are the boundaries of the domain. No change occurs after this point. 
\end{itemize}
	
\noindent The boundary conditions are defined as follows 
\bigskip

\noindent At the inlet ($AB$)
\begin{itemize}
\item $u_r = 0$ and $u_z=5801.9 mm/s$
\item $C_i = 1$ 
\item $T = 300K$
\item $p=1 atmos$
\end{itemize}

\bigskip 
\noindent On $BC$
\begin{itemize}
\item $u_r = 0$ and $u_z=0 mm/s$
\item $\frac{\partial C_i}{\partial r} =0 $ and $\frac{\partial C_i}{\partial z} =0 $
\item $T = 300K$
\item $p=1 atmos$
\end{itemize}

\noindent On $CD$
\begin{itemize}
\item $u_r = 0$ and $u_z=5801.9 mm/s$
\item $ C_{O_2} =21 \% $
\item $T = 300K$
\item $p=1 atmos$
\end{itemize}

\noindent On $DF$ and $FE$
\begin{itemize}
\item $\frac{\partial u_r}{\partial r} =0 $ and $\frac{\partial u_z}{\partial z} =0 $
\item $\frac{\partial C_i}{\partial r} =0 $ and $\frac{\partial C_i}{\partial z} =0 $
\item $\frac{\partial T}{\partial r} =0 $ and $\frac{\partial T}{\partial z} =0 $
\item $\frac{\partial p}{\partial r} =0 $ and $\frac{\partial p}{\partial z} =0 $
\end{itemize}


\noindent On $AF$
\begin{itemize}
\item $\frac{\partial u_r}{\partial r} =0 $ 
\item $\frac{\partial C_i}{\partial r} =0 $ 
\item $\frac{\partial T}{\partial r} =0 $  
\item $\frac{\partial p}{\partial r} =0 $
\end{itemize}








\input{femformulation.tex}


\chapter{Bayesian Statistics}
\section{Review of Theory}

In various sciences and engineering fields, uncertainity quantification problems arise. If we  incorporate data into a model, significant reduction in uncertainty of model prediction is acheived and hence a very important step in many applicaiton. Bayes' formula provides the natural way to do this. Mathematical model governing physical phenomenon often include parameters that need to be determined from experiments by measuring them with the help of devices. The measures values of the parameters have uncertainty in them depending upon the assumptions, noise levels, models and prior knowledge. To estimate the correct values of the parameters we should make assumptions about the noise levels, models and prior knowledge. Let $y$ be the data which is dependent on $u$. 

$$y = G(u)$$. 

\noindent The solution of the inverse problem is the probability distribution of u given y, denoted by $u|y$. The Bayes' formula is given as 

$$P(u|y) = \frac{P(u) P(y|u)}{P(y)}$$ .
 
 \bigskip
 
\noindent Where $P(u)$ is the prior probability distribution of $u$. $P(y|u)$ is the likelihood of $y$ given $u$. $P(y)$ is the scaling factor. 
 \noindent In Bayesian approach, $u$ is treated as random variable with some specified prior probability distribution that incorporates any prior knowledge about $u$ that we believe is true and is independent of the measured data $y$. The result is the posterior probability distribution of $u$ conditional on $y$. The likelihood and prior are selected as per (…........) .

\section{Application to Problem}
\section{Methods of Solution}
\section{Software}



\chapter{Results}

\section{Model 1}

The results displayed in this section are for the
activation energy for the fall off reaction in the ozone
mechanism. The percentage of ozone is taken as 40, 46, 53, 75 and 100
percent according to the experimental data
available~\cite{Streng}. In the
figure~\ref{subfig-1:Raw Chain} and figure~\ref{subfig-2:Histogram} ,  we display results for sample size 1e7 and surrogate
of 1000 points. We show the raw samples generated by MCMC and plot
 the histogram for parameter $E_3$. In the figure~\ref{fig-2:conv_sample} , for
constant surrogate size, the number of samples are changed from 1e5 to
1e7 and convergence is observed. The plot is done for surrogate size of 1000 and raw chain size of $1e5$, $5e5$ , $1e6$, $5e6$ and $1e7$ is taken. In the figure~\ref{fig-3:conv_surrogate},
convergence study is done for surrogates with differing number of
interpolation points. As we increase the number of points in the surrogate, results should be close for different surrogate sizes. The plot is done for surrogate size of 100, 500 and 1000. In this analysis, raw sample chain size is $1e7$. For different starting points of MCMC chain, the MAP point of the resulting pdf does not change. The surrogates for individual
concentrations are constructed using linear interpolation
function. The initial guess for the MAP point is calculated using
Nelder Mead optimization technique. 

\bigskip

In the
figure~\ref{subfig-1:mean_1} and figure~\ref{subfig-2:auto_11},  mean and correlation plots are shown for
the samples of the parameter. The mean plot shows the initial instability due to sum in period of MCMC and after that it remains constant. It shows us that we should be using at least more than these number of samples for our analysis. The figure~\ref{subfig-2:final_e3} shows the posterior of the parameter $E_3$ with the following parameters of the distribution- Mean:  28.3, Std. Dev.:  0.78, Skewness:  0.066 and Kurtosis:  0.035.

\bigskip

 It is necessary to ensure that the samples of the parameter which we are drawing are fitting the flamespeed values of the experiment. In the figure~\ref{subfig-1:40_1}, figure~\ref{subfig-2:46_1}, figure~\ref{subfig-3:53_1}, figure~\ref{subfig-4:75_1}, and figure~\ref{subfig-5:100_1},   we calculate the flamespeed for all the parameters drawn using the surrogate generated before for 40$\%$, 46$\%$, 53$\%$, 75$\%$, and 100 $\%$ respectively. We have taken $1e7$ sample size and calculated flamespeed for all the concentrations of ozone mentioned. We have a good fit to the data overall.



%\subsection{Convergence Study: Number of Samples }

% In this section, we see the convergence of the probability distribution as we increase the raw chain sample size. The plot is done for surrogate size of 1000. In this analysis, raw chain size of $1e5$, $5e5$ , $1e6$, $5e6$ and $1e7$ is taken.



%\subsection{Convergence Study: Surrogate }

 %In this section, we see the convergence of the surrogate. As we increase the number of points in the surrogate, results should be close for different surrogate sizes. The plot is done for surrogate size of 100, 500 and 1000. In this analysis, raw sample chain size is $1e7$.



%\subsection{Flamespeed data fit}

 %It is necessary to ensure that the samples of the parameter which we are drawing are fitting the flamespeed values of the experiment. In this section, we calculate the flamespeed for all the parameters drawn using the surrogate generated before. We have taken $1e7$ sample size and calculated flamespeed for different concentrations of ozone.

 


%\subsection{Mean and Autocorrelation Plots}

% In this section, we show the mean of the samples and autocorrelation plots . The mean plot shows the initial instability due to sum in period of MCMC and after that it remains constant. It shows us that we should be using at least more than these number of samples for our analysis. The last figure shows the histogram and the various parameters of the distribution are Mean:  28.3, Std. Dev.:  0.78, Skewness:  0.066 and Kurtosis:  0.035.






\begin{figure}[H]
\subfloat[MCMC raw chain of samples \label{subfig-1:Raw Chain}]{%
     \includegraphics[scale=0.45]{model_1/simple_ip_chain_pos_filt}
    }
\subfloat[Histogram\label{subfig-2:Histogram}]{%
     \includegraphics[scale=0.47]{model_1/simple_ip_hist_raw}
    }
\caption{MCMC raw chain and histogram}
    \end{figure}
%
  \begin{figure}[H]
  \ContinuedFloat
  \centering
\subfloat[KDE \label{subfig-4:KDE_1}]{
        \includegraphics[scale=0.7]{model_1/simple_ip_kde_raw}
            }
\caption{ Parameter  kernel density estimation for $E_3$}
\end{figure}


\begin{figure}[H]
\includegraphics[scale = 0.5]{model_1/sample_conv}
    \caption{Convergence for surrogate size 1000}
    \label{fig-2:conv_sample}
\end{figure}



\begin{figure}[H]
\includegraphics[scale=0.5]{model_1/conv_surrogate}
    \caption{Convergence for surrogate size 100, 500 and 1000}
    \label{fig-3:conv_surrogate}
\end{figure}


\begin{figure}[H]
   \subfloat[ Flame speed for 40 \% ozone \label{subfig-1:40_1}]{
        \includegraphics[scale=0.45]{model_1/flame_40.pdf}
       }
\subfloat[Flame speed for 46 \% ozone \label{subfig-2:46_1}]{
        \includegraphics[scale=0.45]{model_1/flame_46.pdf}
            }
\end{figure}


 \begin{figure}[H]
  \ContinuedFloat
   \subfloat[ Flame speed for 53 \% ozone \label{subfig-3:53_1}]{
        \includegraphics[scale=0.45]{model_1/flame_53.pdf}
       }
\subfloat[Flame speed for 75 \% ozone \label{subfig-4:75_1}]{
        \includegraphics[scale=0.45]{model_1/flame_75.pdf}
            }
\end{figure}


 \begin{figure}[H]
  \ContinuedFloat
   \subfloat[ Flame speed for 100 \% ozone \label{subfig-5:100_1}]{
        \includegraphics[scale=0.45]{model_1/flame_100.pdf}
       }
  \caption{Flamespeed Data Fit}
\end{figure}


 \begin{figure}[H]
  \ContinuedFloat
  \centering
   \subfloat[ Mean  \label{subfig-1:mean_1}]{
        \includegraphics[scale=0.7]{model_1/M1_running_avg.pdf}
       }
     \quad
\subfloat[Autocorrelation  \label{subfig-2:auto_11}]{
        \includegraphics[scale=0.7]{model_1/M1_autocorr.pdf}
            }
            \caption{Mean and autocorrelation for sample size 1e7}
			\end{figure}
 \begin{figure}[H]
  \centering
\subfloat[ \label{subfig-2:final_e3}]{
        \includegraphics[scale=0.7]{model_1/E3.pdf}
            }
            \caption{$E_3$ posterior distribution}
\end{figure}






\chapter{Conclusion}
%\begin{ubbackmatter}
\bibliographystyle{plain}
\references[Bibliography]{references}
%\bibliography{references}
%\end{ubbackmatter}
\end{document}
\endinput
%%
%% End of file `thesis.tex'.
