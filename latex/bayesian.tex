\chapter{Bayesian Methods}
\section{Review of Theory}

The growth of computations technology has enabled uncertainity quantification for complex engineering problems. If we  incorporate data into a model, significant reduction in uncertainty of model prediction is acheived and hence a very important step in many applicaiton. Bayes' formula provides the natural way to infer the necessary uncertainty information. Mathematical model governing physical phenomenon often include parameters that need to be determined from experiments by measuring them with the help of devices. The measured values of the parameters have uncertainty in them depending upon the assumptions, noise levels, models and prior knowledge. To estimate the correct values of the parameters we should make assumptions about the noise levels, models and prior knowledge. Let $y$ be the data which is dependent on independent parameter $u$. 

$$y = G(u)$$. 

\noindent The solution of the inverse problem is the probability distribution of u given y, denoted by $u|y$. The Bayes' formula is given as 

$$P(u|y) = \frac{P(u) P(y|u)}{P(y)}$$ .
 
 \bigskip
 
\noindent Where $P(u)$ is the prior probability distribution of $u$. $P(y|u)$ is the likelihood of $y$ given $u$. $P(y)$ is the scaling factor. 
 \noindent In Bayesian approach, $u$ is treated as random variable with some specified prior probability distribution that incorporates any prior knowledge about $u$ that we believe is true and is independent of the measured data $y$. The result is the posterior probability distribution of $u$ conditional on $y$.

\section{Application to Problem}

\noindent We have the experimental data for flamespeed from Streng \cite{Streng}. Our goal is to do uncertainty quantification for chemical parameters namely Activation Energy $ E $ and pre exponential factor $ A $ for all the three reactions involved in the ozone oxygen combustion model. We first did sensitivity analysis for flamespeed based on all the chemical parameters. We found that the activation energy and the pre exponential factor for the first reaction of the mechanism did not have significant effect on the flamespeed. Hence moving further, we would not consider these parameters as uncertain. The remaining 4 parameters are considered uncertain. 

\noindent Let us denote flamespeed by $V_f$, the activation energy for second and third reaction for mechanism as $E_2$ and $E_3$ respectively and the pre exponential factor as $A_2$ and $A_3$ recpectively. We model our uncertain parameter as continous random variable. By bayes theorem, the mathematical equation for uncertainty quantification for chemical parameters is written as follows. 

$$P(E_1, E_2, A_2, A_3 |V_f ) = \frac{P(V_f|E_1, E_2, A_2, A_3) P(E_1)*P(E_2)*P(A_2)*P(A_3)}{\int_x P(V_f|E_1, E_2, A_2, A_3)* P(E_1)*P(E_2)*P(A_2)*P(A_3) }$$ 

\noindent $P(E_1, E_2, A_2, A_3 |V_f ) $ is the resulting posterior distribution for the uncertain parameters in the model. $P(V_f|E_1, E_2, A_2, A_3)$ is called the likelihood distribution of the data given the parameters and $ P(E_1), P(E_2) , P(A_2), P(A_3)$ are the prior distribution of the parameter. 

\noindent In our model, we have selected uniform priors for the parameter. This means that given the range of parameter i.e from a to b, the probability of selcting a parameter is same for all the values in this range. The continuous pdf is given as 

$$f(x) = \frac{1}{b -a}$$


\noindent We have selected likelihood as gaussian with the experimental data as mean and standard deviation of 10 percent of the data. The resulting pdf is given as 

$$f(x) = \frac{1}{{\sigma \sqrt {2\pi } }}e^{{{ - \left( {x - \mu } \right)^2 } \mathord{\left/ {\vphantom {{ - \left( {x - \mu } \right)^2 } {2\sigma ^2 }}} \right. \kern-\nulldelimiterspace} {2\sigma ^2 }}} $$

\noindent $\int_x P(V_f|E_1, E_2, A_2, A_3)* P(E_1)*P(E_2)*P(A_2)*P(A_3)$ is the scaling factor. 

\section{Methods of Solution}

\noindent We use markov chain monte carlo method for estimating the posterior distribution by simulations. Markov chain is formed from successive random selections, the stationary distribution of which is the target distribution. In the Metropolis–Hastings algorithm, samples are selected from an arbitrary “proposal” distribution and are retained or not according to an acceptance rule. 

\section{Software}