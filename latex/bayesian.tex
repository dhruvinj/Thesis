\chapter{Bayesian Methods}\label{chap:bayes}
\section{Review of Theory}

The growth of computations technology has enabled uncertainity
quantification for complex engineering problems. If we incorporate the
inevitable uncertainty present in experimental
data into the calibration of our models, the quantification of
uncertainty of complex models can be
achieved. Bayes' formula provides a natural foundation on which to
base such a calibration procedure.
%% Mathematical model governing
%% physical phenomenon often include parameters that need to be
%% determined from experiments.
To estimate the model parameters, we must make
assumptions about the noise levels in the data, models, and
encapsulate our prior knowledge. Let
$\mathbf{y}$ be the data which is implicitly dependent on the parameters parameter $\mathbf{m}$.
%
\begin{equation}
  \mathbf{y} = G(\mathbf{m})
\end{equation}
%
  In the Bayesian approach, $\mathbf{m}$ is treated as random variable with some
  specified prior probability distribution that incorporates any prior
  knowledge about $\mathbf{m}$ that we believe is true. The solution of the inverse problem is the
  probability distribution of $\mathbf{m}$ given $\mathbf{m}$, denoted by $P(\mathbf{m}|\mathbf{y})$. Then, Bayes'
  formula is given as~\cite{Sivia}
  %
  \begin{equation}
    P(\mathbf{m}|\mathbf{y}) = \frac{P(\mathbf{m}) P(\mathbf{y}|\mathbf{m})}{P(\mathbf{y})}
  \end{equation}
%
 where $P(\mathbf{m})$ is the prior probability distribution, $P(\mathbf{y}|\mathbf{m})$
 is the likelihood distribution, and $P(\mathbf{y})$ is the scaling factor.
 The result is the posterior probability distribution of $P(\mathbf{m}|\mathbf{y})$.

\section{Methods of Solution}

In only the simplest of circumstances is an analytical solution
possible. Otherwise, we must resort to numerical solution.
 We use Markov Chain Monte Carlo (MCMC) methods for estimating the posterior
 distribution by repeated sampling of our forward model. In the
 present work, the forward model is a numerical simulation of
 one-dimensional flames. A Markov Chain is formed from successive
 random selections, the stationary distribution of which is the target
 distribution. In the Metropolis–Hastings algorithm, samples are
 selected from an arbitrary “proposal” distribution and are retained
 or not according to an acceptance rule~\cite{RobertCasella2004}.

 Furthermore, because of the heavy sampling burden, even ``fast''
 models may be challening to leverage within an MCMC sampler. Thus,
 one often resorts to so-called surrogate models where instead of
 computing with the ``full model'' directly, we use a surrogate that
 is constructed ahead of time, but that can be evaluated very
 quickly. For the present work, we use simple linear Lagrange
 interpolation over the parameter space. While this strategy is not
 effective in high parameter dimensions, due the rapid growth of the
 number of function evaluations required, for small dimensional
 problems, the strategy is easy and effective.

 We heavily leverage the QUESO C++ software package~\cite{QUESO} for the
 development of the the statistical surrogate as well as the MCMC
 sampling. In particular, we use Metropolis-Hastings MCMC; the QUESO
 library deploys the so-called Delayed Rejection Adaptive Monte Carlo
 (DRAM) sampling strategy~\cite{DRAM}. Our interface consists of a
 simple wrapper class around the Cantera solver for representing the
 $G(\mathbf{m})$ map for which we can then easily plugin to the array
 of algorithms available in QUESO, including the surrogate
 construction and the MCMC sampling.

\section{Application to Ozone Combustion}

 We have the experimental data for flamespeed from Streng
 \cite{Streng}, as discussed in Section~\ref{sec:ozone-data}.
 Our goal is the inference of
 chemical parameters for the Ozone mechanism described in
 Equation~\eqref{eq:ozone-mech}. For this mechanism, each of the
 reaction rates is of the Arrhenius form described in
 Equation~\eqref{eq:arrhenius}.
 Thus, the parameters of interest are the
 activation energy $E$ and the
 pre-exponential factor $A$ for all three reactions involved in the
 ozone combustion model. Since we are only considering variation in
 the chemical kinetics, we are implicitly assuming that all other
 approximations are valid, namely the conservation, transport, and
 thermodynamic models. Further, we are assuming adequate numerical
 resolution of the solutions and convergence of the iterative methods.

 First, a rudimentary sensitivity analysis was performed to identify
 those parameters to which the laminar flame speed, denoted $V_f$, was most
 sensitive. Then, based on those results, we construct several
 statistical models on which we perform Bayesian inference. That is,
 each statistical model may consider differing priors, likelihoods,
 and uncertain parameters; however, the experimental data is the
 same. Those parameters that are not treated as uncertain are held
 fixed at nominal values.

 In all cases,
the prior is uniform, which differing bounds for each model, and the
likelihood is assumed Gaussinan. In~\cite{Streng}, no indication is
given on the error associated with each of the measured laminar flame
speeds. Thus, we assume that the error is zero mean with a standard
deviation 10\% of the data value for each of the reported laminar
flame speeds. Further, we assume that each of the measurements is
independent. Therefore, the likelihood function will take the form
%
\begin{equation}
P(\mathbf{y}|\mathbf{m}) = \frac{1}{(2\pi)^d \sqrt{|\Sigma|}}
\exp\left[ -\frac{1}{2}\left(G(\mathbf{m}) -
  \mathbf{y}\right)^T\Sigma^{-1} \left(G(\mathbf{m}) -
  \mathbf{y}\right) \right]
\end{equation}
%
where $d$ is the dimension of the data and the covariance matrix, $\Sigma$, is diagonal.
We use the five data points reported in Table~\ref{tab:ozone-flame-data}.
Thus, in all cases, $d = 5$ and the vector $\mathbf{y} \in \mathbb{R}^5$.

 \section{Sensitivity study}

\noindent First we started with sensivity study of the chemical parameters. We change the values of pre exponential factor and activation energy for all three reactions by $\pm$ 100 $\% $ and run our forward model and note the values of flamespeed obtained. We repeat this activity for 40, 46, 53, 75 and 100 $\%$ ozone cases. For all the concentrations, it is seen that the variation of flamespeed with around 0.5 $\%$ for the pre exponential and activation energy got first reaction of the mechanism. The variation of pre expoenntial factor of the second reaction and thrid reactions for all the concentrations has significant change of flamespeed values. The most sensitive parameters among all the four parameters is the activation energy for the thrid reaction of the mechanism where the change of more than $\pm 200\%$ is observed in flame speed. 

\noindent Going further we would not consider the uncertainties involved in the chemical parameters of the first reaction. We would consider the remaining four parameters for our study.  

 %% We first did sensitivity analysis for
 %% flamespeed based on all the chemical parameters. We found that the
 %% activation energy and the pre exponential factor for the first
 %% reaction of the mechanism did not have significant effect on the
 %% flamespeed. Hence moving further, we would not consider these
 %% parameters as uncertain. The remaining 4 parameters i.e Activation
 %% Energy and pre exponential factors for second and third reactions
 %% were highly sensitive to the flame speed calcualtions. The Acivation
 %% energy for the third reaction was most sensitive. We will approach to
 %% the uncertainty quantification problem step by step with different
 %% models.

 %% Let us denote flamespeed by $V_f$, the activation energy for second
 %% and third reaction for mechanism as $E_2$ and $E_3$ respectively and
 %% the pre exponential factor as $A_2$ and $A_3$ recpectively.


\subsection{Model 1}
 In this model, we perform inference on the most
 sensitive parameter of all the parameters, i.e activation energy for
 the third reaction, $E_3$. We take our prior as uniform with domain
 bounds of -10 to 40 \todo{UNITS!}. Thus, for Model 1, the vector
 $\mathbf{m} \in \mathbb{R}$ and our mapping $G(\mathbf{m}):
 \mathbb{R} \rightarrow \mathbb{R}^5$.


\subsection{Model 2}


 In this model, we consider two parameters: the activation energy for
 the third reaction, $E_3$,
 and the activation energy for the second reaction, $E_2$. We take our priors
 as uniform with domain bounds of -10 to 40 \todo{UNITS!} for both of the
 parameters. Thus, for Model 2, the vector
 $\mathbf{m} \in \mathbb{R}^2$ and our mapping $G(\mathbf{m}):
 \mathbb{R}^2 \rightarrow \mathbb{R}^5$.


 %% The likelihood is gaussian with experimental data values
%%  from Streng\cite{Streng} as mean standard deviation of 10 \% of the
%%  data value. The probability distributions functions are given as

%%  The prior distribution for each parameter as all points x in its
%%  respective domain (a,b) is

%% $$f(x) = \frac{1}{b -a}$$


%%  Since the likelihood as gaussian. The resulting pdf is given as

%% $$f(x) = \frac{1}{{\sigma \sqrt {2\pi } }}e^{{{ - \left( {x - \mu }
%%        \right)^2 } \mathord{\left/ {\vphantom {{ - \left( {x - \mu }
%%              \right)^2 } {2\sigma ^2 }}}
%%        \right. \kern-\nulldelimiterspace} {2\sigma ^2 }}} $$


%%  Where $\mu$ is the data value. Since we have 7 data values
%%  (neglecting the 17 \% ozone case due to numerical difficulty), we
%%  will have 7 gaussian function and the resulting likelihood function
%%  will be product of all the gaussian pdfs. We model our uncertain
%%  parameter as continous random variable. By bayes theorem, the
%%  mathematical equation for uncertainty quantification for chemical
%%  parameters is written as follows.

%%  $$P( E_3, E_2 |V_f ) = \frac{P(V_f| E_2,E_3)* P(E_2)* P(E_3)}{\int_x
%%    P(V_f|E_2,E_3)* P(E_2)*P(E_3)}$$

%%   Where $P(E_3,E_2 |V_f)$ is the resulting posterior distribution of
%%   $E_3$ and $E_2$ given flamespeed $V_f$ data, $P(V_f| E_2 ,E_3)$ is
%%   the likelihood which is product of the probabilties of each of the
%%   gaussian pdfs of different data points, and $ P(E_3)$ and $P(E_2)$
%%   is the prior distribution of the parameter, and $\int_x
%%   P(V_f|E_2,E_3)* P(E_2)*P(E_3)$ is the scaling factor.


\subsection{Model 3}


 In this model, we consider two parameters: the activation energy
 for the third reaction, $E_3$, and the pre-exponential factor for the
 third reaction, $A_3$.
 We take our priors
 as uniform with domain bounds of -10 to 40\todo{UNITS!} for $E_3$ and
 1.4e+1\todo{IS THIS NUMBER RIGHT?} to
 1.4e+22\todo{UNITS!} for $A_3$ parameter. \todo{AREN'T YOU TAKING THE
   LOG OF $A_3$?}
Thus, for Model 3, the vector
 $\mathbf{m} \in \mathbb{R}^2$ and our mapping $G(\mathbf{m}):
\mathbb{R}^2 \rightarrow \mathbb{R}^5$.

 %% The likelihood is gaussian with
%%  experimental data values from Streng\cite{Streng} as mean standard
%%  deviation of 10 \% of the data value. The probability distributions
%%  functions are given as

%%  The prior distribution for each parameter as all points x in its
%%  respective domain (a,b) is

%% $$f(x) = \frac{1}{b -a}$$


%%  Since the likelihood as gaussian. The resulting pdf is given as

%% $$f(x) = \frac{1}{{\sigma \sqrt {2\pi } }}e^{{{ - \left( {x - \mu }
%%        \right)^2 } \mathord{\left/ {\vphantom {{ - \left( {x - \mu }
%%              \right)^2 } {2\sigma ^2 }}}
%%        \right. \kern-\nulldelimiterspace} {2\sigma ^2 }}} $$


%%  Where $\mu$ is the data value. Since we have 7 data values
%%  (neglecting the 17 \% ozone case due to numerical difficulty), we
%%  will have 7 gaussian function and the resulting likelihood function
%%  will be product of all the gaussian pdfs. We model our uncertain
%%  parameter as continous random variable. By bayes theorem, the
%%  mathematical equation for uncertainty quantification for chemical
%%  parameters is written as follows.

%%  $$P( A_3, E_3 |V_f ) = \frac{P(V_f| E_3,A_3)* P(E_3)* P(A_3)}{\int_x
%%    P(V_f|E_3,A_3)* P(E_3)*P(A_3)}$$

%%   Where $P(E_3,A_3 |V_f)$ is the resulting posterior distribution of
%%   $A_3$ and $E_3$ given flamespeed $V_f$ data, $P(V_f| E_3 ,A_3)$ is
%%   the likelihood which is product of the probabilties of each of the
%%   gaussian pdfs of different data points, and $ P(E_3)$ and $P(A_3)$
%%   is the prior distribution of the parameter, and $\int_x
%%   P(V_f|E_3,A_3)*P(E_3)*P(A_3)$ is the scaling factor.


\subsection{Model 4}


 In this model, we consider three parameters: activation energy for
 the third reaction, $E_3$,
 activation energy for the second reaction, $E_2$, and the pre-exponential
 factor for the third reaction, $A_3$. We take our priors as uniform
 with domain bounds of -10 to 40 for $E_2$, -10 to 40 for $E_3$ and
 1.4e+1 to 1.4e+22 for $A_3$ parameter. \todo{UNITS? LOG of $A_3$?}
Thus, for Model 4, the vector
 $\mathbf{m} \in \mathbb{R}^3$ and our mapping $G(\mathbf{m}):
\mathbb{R}^3 \rightarrow \mathbb{R}^5$.

%% The likelihood is gaussian
%%  with experimental data values from Streng\cite{Streng} as mean
%%  standard deviation of 10 \% of the data value. The probability
%%  distributions functions are given as

%%  The prior distribution for each parameter as all points x in its
%%  respective domain (a,b) is

%% $$f(x) = \frac{1}{b -a}$$


%%  Since the likelihood as gaussian. The resulting pdf is given as

%% $$f(x) = \frac{1}{{\sigma \sqrt {2\pi } }}e^{{{ - \left( {x - \mu }
%%        \right)^2 } \mathord{\left/ {\vphantom {{ - \left( {x - \mu }
%%              \right)^2 } {2\sigma ^2 }}}
%%        \right. \kern-\nulldelimiterspace} {2\sigma ^2 }}} $$


%%  Where $\mu$ is the data value. Since we have 7 data values
%%  (neglecting the 17 \% ozone case due to numerical difficulty), we
%%  will have 7 gaussian function and the resulting likelihood function
%%  will be product of all the gaussian pdfs. We model our uncertain
%%  parameter as continous random variable. By bayes theorem, the
%%  mathematical equation for uncertainty quantification for chemical
%%  parameters is written as follows.

%%  $$P( A_3, E_3,E_2 |V_f ) = \frac{P(V_f| E_2,E_3,A_3)* P(E_2)*P(E_3)*
%%    P(A_3)}{\int_x P(V_f|E_2,E_3,A_3)* P(E_2)*P(E_3)*P(A_3)}$$

%%   Where $P(E_3,A_3,E_2 |V_f)$ is the resulting posterior distribution
%%   of $A_3$, $E_3$, and $E_2$ given flamespeed $V_f$ data, $P(V_f| E_2,
%%   E_3 ,A_3)$ is the likelihood which is product of the probabilties of
%%   each of the gaussian pdfs of different data points, and $P(E_3)$,
%%   $P(E_2)$, and $P(A_3)$ is the prior distribution of the parameter,
%%   and $\int_x P(V_f|E_2,E_3,A_3)* P(E_2)*P(E_3)*P(A_3)$ is the scaling
%%   factor.
