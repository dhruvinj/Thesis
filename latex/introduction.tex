\chapter{Introduction}

It is important to understand the dynamics of reacting flows as it is inherent in number of areas of science and engineering such as combustion, surface chemistry and reentry flows . To understand these systems in detail, it is necessary to use numerical simulations. These simulations require simultaneous numerical resolution of chemical reactions; diffusive transport and fluid mechanics. The combination of these three factors make simulations demanding of computational resources. In the deflagration regime where the burning speed also known as laminar flame speed and associated velocity scales are much smaller than the speed of sound in the fluid, the problem is acute.

\bigskip
\noindent For low speed combustion flows, Low Mach number asysmptotics of the flow equations exploit the inherent separation of scales in such systems by analytically eliminating acoustic wave propagation entirely from the dynamics, while preserving the important compressibility effects arising from reactions and transport.

\bigskip
\noindent  We need models that provide predictive and explanatory power. These models should be complex enough to explain the observed phenomena. Also, these models should be simple enough to generalise to future observations. Bayesian inference provides a systematic framework to infer such models from the observed data. 
In this work, We aim at doing uncertainty quantification for chemical parameters in the ozone oxygen combustion model. Experiments\cite{Streng} have successfully measured the flamespeed for premixed ozone oxygen   fuel at various concentrations starting from  17 percent ozone to 100 percent ozone. We take this data in our model and try to infer the chemical parameters that would fit this data. In bayesian approach, we take into consider our prior knowledge about the parameters and the likelihood of observing the data given the parameters. The resulting answer id the probability distribution of the parameters given the data.  


\bigskip
\noindent A.G Streng and A.V. Grosse \cite{Streng}  studied the ozone oxygen flame experimentally. The stability of ozone and the rates of decomposition or explosion were investigated by Armour research foundation. Ozone was burned to oxygen from a simple burner tip in the range from 17 percent to 100 percent initial concentration of ozone in the mixture. The flame temperatures were calculated from enthalpy data and dissociation constants of oxygen using kelley's tables. The concentration of ozone was kept constant with error of   0.2 percent. Two methods were used to determine burning velocity i.e open tube method and the burning tip method. We will be using experimental burning velocity of the burning tip method. The burner tip experiments were carried out in standard apparatus, using pyrex glass aluminium tips with an inner diameter of 3 to 0.65mm. The flames were readily observed by the standard schlieren method at all concentrations above 30 mole percent. The measurements were all carried out in the laminar flow region and the reynolds number of the flow was below 2000. The initial conditions are 300K temperature and 1.0 atmosphere pressure. The results of burning velocities of ozone flames were compared with theoretical burning velocities of Dr. Von Karman and his associates. They were found to be in close agreements. 
