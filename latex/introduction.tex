\chapter{Introduction}

It is important to understand deeply the dynamics of reacting flows as it is inherent in number of areas of science and engineering like combustion, surface chemistry etc . To understand these systems in detail, large simulations are necessary. It also requires simultaneous numerical resolution of chemical reactions, diffusive
transport and fluid mechanics. The combination of three factor make simulations some of the most demanding in the area of combustion. In the deflagration (combustion wave that propagates through a gas or across the surface of an explosive at subsonic speeds, driven by the transfer of heat) regime where the burning speed (also known as laminar flame speed) and associated velocity scales are much smaller than the speed of sound in the fluid, the problem is acute.

\bigskip
\noindent For low speed combustion flows, Low Mach number asysmptotics of the flow equations exploit the inherent separation of scales in such systems by analytically eliminating acoustic wave propagation entirely from the dynamics, while preserving the important compressibility effects arising from reactions and transport.
\section{Literature Review}
It is shown by Muller that a multiple-time scale, single-space scale asymptotic analysis of the compressible Navier-Stokes equations reveals that the zeroth-order global thermodynamic pressure, the divergence of velocity and the material change of density are affected by heat-release rate and heat conduction at low-Mach-numbers.

\bigskip 
\noindent His result show that the acoustic time change of the heat-release rate as the dominant source of sound in low-Mach-number flow. The asymptotic expansion of all flow variables show that the viscous and buoyancy forces enter the computation of the second-order incompressible pressure in low-Mach-number flow in a similar way as they enter the pressure computation in incompressible flow, except that the
velocity-divergence constraint is non zero. He averaged flow equation over an acoustic wave period, the averaged velocity tensor described the net acoustic effect on the averaged flow field. Once acoustics were removed from the equations altogether it lead to the low-Mach-number equations.

\bigskip
\noindent As many flows of interest can be considered as incompressible. The incompressibility assumption makes the problem simpler than if a full compressible flow is considered. Codine  proposed that for ideal fluids, with isentropic condition, solutions of the incompressible Navier Stokes equations can be found as the limit of solutions of the compressible ones as the Mach number tends to zero under certain assumptions on the initial data.

\noindent In her paper she showed that small Mach number limit gives rise to a separation of the pressure into a constant-in-space thermodynamic pressure and a mechanical pressure that has to be used in the momentum equation. This leads to a removal of the acoustic modes and the flow behaves as incompressible, in the sense that the mechanical pressure is determined by the mass conservation
equation and not by the state equation. However, large variations of density
due to temperature variations are allowed. She showed that when the Mach number is small the hyperbolic wave equation for the pressure becomes an elliptic
equation for the first order pressure p(2), thus showing the implicit (“incompressible” or “mechanical”) character of this pressure component

\bigskip
\noindent A.G Streng and A.V. Grosse  studied the ozone oxygen flame experimentally. The stability of ozone and the rates of decomposition or explosion were investigated by Armour research foundation. Ozone was burned to oxygen from a simple burner tip in the range from 17 percent to 100 percent initial concentration of ozone in the mixture. The flame temperatures were calculated from enthalpy data and dissociation constants of oxygen using kelley's tables. The concentration of ozone was kept constant with error of   0.2 percent. Two methods were used to determine burning velocity i.e open tube method and the burning tip method. We will be using experimental burning velocity of the burning tip method. The burner tip experiments were carried out in standard apparatus, using pyrex glass aluminium tips with an inner diameter of 3 to 0.65mm. The flames were readily observed by the standard schlieren method at all concentrations above 30 mole percent. The measurements were all carried out in the laminar flow region and the reynolds number of the flow was below 2000. The initial conditions are 300K temperature and 1.0 atmosphere pressure. The results of burning velocities of ozone flames were compared with theoretical burning velocities of Dr. Von Karman and his associates. They were found to be in close agreements. 