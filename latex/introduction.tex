\chapter{Introduction}

\section{Motivation}
It is important to understand the dynamics of reacting flows as it is inherent in number of areas of science and engineering such as combustion, surface chemistry, and reentry flows . To understand these systems in detail, it is necessary to use numerical simulations. These simulations require simultaneous numerical resolution of chemical reactions, diffusive transport, and fluid mechanics. The combination of these three factors make simulations demanding of computational resources. The problem is acute in the deflagration regime where the burning speed, also known as laminar flame speed and associated velocity scales are much smaller than the speed of sound in the fluid.

\bigskip
\noindent For low speed combustion flows, low Mach number asysmptotics of the flow equations exploit the inherent separation of scales in such systems by analytically eliminating acoustic wave propagation entirely from the dynamics, while preserving the important compressibility effects arising from reactions and transport.

\bigskip
\noindent  We need models that provide predictive and explanatory power. These models should be complex enough to explain the observed phenomena. Also, these models should be simple enough to generalize to future observations. These models inevitably contain unknown parameters that must be caliberated against experimental data. Traditionally, optimization approaches have been used. However, such a deterministic approaches fail to incorporate information about the uncertainty in the data. Bayesian inference provides a systematic framework to infer such parameters from the observed data and in particular provides a systematic approach to incorporate uncertainty. 
In this work, We aim at performing Bayesian inference for chemical parameters in an ozone combustion model. Experiments\cite{Streng} have successfully measured the flamespeed for premixed ozone oxygen   fuel at various concentrations starting from  17 percent ozone to 100 percent ozone. We use this data in our model and try to infer the chemical parameters that would fit this data. In Bayesian approach, we take into consideration our prior knowledge about the parameters and the likelihood of observing the data given the parameters. The resulting answer is the probability distribution of the parameters given the data.  


\bigskip
\noindent In this work, we begin with mathematical model for the reacting flows in the first section. In this section, we use 1D combustion model from the principles of combustion by kuo. We describe the transport model which we we are using for the calculations of kinetic rates, viscocity, and diffusion rates. In next section we do Bayesian inference of the chemical parameters. We describe different models used by us during the entire analysis. We also describe the method of solutions adopted by us for solving the inverse problem. In secton 4, we plot the results of the inverse problem for the models decribed in section 3. In secton 5, we develop finite element model for 2D premixed laminar flames. Here we describe about the stabilization techniques used and lay the foundation for future Bayesian inference for the chemical parameters for 2D flames.  


\section{Literature review}

It is shown by Muller~\cite{Muller} that a multiple-time scale, single-space scale asymptotic analysis of the compressible Navier-Stokes equations reveals that the zeroth-order global thermodynamic pressure, the divergence of velocity; and the material change of density are affected by heat-release rate and heat conduction at low-Mach-numbers.

\bigskip 
\noindent His result show that the acoustic time change of the heat-release rate as the dominant source of sound in low-Mach-number flow. The asymptotic expansion of all flow variables show that the viscous and buoyancy forces enter the computation of the second-order incompressible pressure in low-Mach-number flow in a similar way as they enter the pressure computation in incompressible flow, except that the
velocity-divergence constraint is non zero. He averaged flow equation over an acoustic wave period, the averaged velocity tensor described the net acoustic effect on the averaged flow field. Once acoustics were removed from the equations altogether it lead to the low-Mach-number equations.

\bigskip
\noindent The incompressibility assumption makes the problem simpler than if a full compressible flow is considered. Codine\cite{Codina} proposed that for ideal fluids, with isentropic condition, solutions of the incompressible Navier Stokes equations can be found as the limit of solutions of the compressible ones as the Mach number tends to zero under certain assumptions on the initial data.

\noindent In this paper, they showed that small Mach number limit gives rise to a separation of the pressure into a constant-in-space thermodynamic pressure and a mechanical pressure that has to be used in the momentum equation. This leads to a removal of the acoustic modes and the flow behaves as incompressible, in the sense that the mechanical pressure is determined by the mass conservation
equation and not by the state equation. However, large variations of density
due to temperature variations are allowed. They showed that when the Mach number is small the hyperbolic wave equation for the pressure becomes an elliptic
equation for the first order pressure p(2), thus showing the implicit (“incompressible” or “mechanical”) character of this pressure component.
