\chapter{Appendix}\label{chap:app}

 This works shows the asymptotic anaylsis of Navier stokes equation which was developed by Muller\cite{Muller}. It is shown by Muller~\cite{Muller} that a multiple-time scale, single-space scale asymptotic analysis of the compressible Navier-Stokes equations reveals that the zeroth-order global thermodynamic pressure, the divergence of velocity; and the material change of density are affected by heat-release rate and heat conduction at low-Mach-numbers.
It is shown by Muller~\cite{Muller} that a multiple-time scale, single-space scale asymptotic analysis of the compressible Navier-Stokes equations reveals that the zeroth-order global thermodynamic pressure, the divergence of velocity; and the material change of density are affected by heat-release rate and heat conduction at low-Mach-numbers.

\bigskip
 His result show that the acoustic time change of the heat-release rate as the dominant source of sound in low-Mach-number flow. The asymptotic expansion of all flow variables show that the viscous and buoyancy forces enter the computation of the second-order incompressible pressure in low-Mach-number flow in a similar way as they enter the pressure computation in incompressible flow, except that the
velocity-divergence constraint is non zero. He averaged flow equation over an acoustic wave period, the averaged velocity tensor described the net acoustic effect on the averaged flow field. Once acoustics were removed from the equations altogether it lead to the low-Mach-number equations.

\bigskip
 The incompressibility assumption makes the problem simpler than if a full compressible flow is considered. Codine  proposed that for ideal fluids, with isentropic condition, solutions of the incompressible Navier Stokes equations can be found as the limit of solutions of the compressible ones as the Mach number tends to zero under certain assumptions on the initial data.

 In this paper, they showed that small Mach number limit gives rise to a separation of the pressure into a constant-in-space thermodynamic pressure and a mechanical pressure that has to be used in the momentum equation. This leads to a removal of the acoustic modes and the flow behaves as incompressible, in the sense that the mechanical pressure is determined by the mass conservation
equation and not by the state equation. However, large variations of density
due to temperature variations are allowed. They showed that when the Mach number is small the hyperbolic wave equation for the pressure becomes an elliptic
equation for the first order pressure p(2), thus showing the implicit (“incompressible” or “mechanical”) character of this pressure component.

\section{Asymptotic Analysis}
\subsection{Governing Equations}
The flow of a compressible fluid is described in terms of the velocity $(u^*)$, pressure $(p^*)$, density $(\rho^*)$, and temperature$(T^*)$ fields. These fields are solutions of the compressible Navier Stokes equations that describe the dynamics of the system
and that are statements of conservation of mass, momentum and energy and a state equation relating the thermodynamic variables.  

The system of equation that needs to be solved reads
\begin{eqnarray}
\frac{D \rho^*}{D t^*}\hspace{1mm} \hspace{1mm} &=& - \rho^* \hspace{1mm}\nabla \cdot u^* \\
\rho^* \frac{D u^*}{D t} \hspace{1mm} &=& -\hspace{1mm}\nabla p^*\hspace{1mm}+\hspace{1mm}\nabla \cdot \tau^* + \rho^* g^* \\
\rho^* C_p^* \frac{D T^*}{D t^*} \hspace{1mm}&=&\hspace{1mm} \tau^* \nabla \cdot u^* \hspace{1mm}- \hspace{1mm}\nabla \cdot q^* + \hspace{1mm} \beta^* T^* \frac{D p^*}{D t^*}\hspace{1mm}\\
\frac{D \rho^*}{D t^*} &=& \hspace{1mm} - \beta^* T^* \frac{D \rho^*}{D t^*} + \alpha^* \rho^* \hspace{1mm}\frac{D p^*}{D t^*}
\end{eqnarray}

Where $u^*$ is the velocity vector,$\rho^*$ is the density,$\beta^*$ is the thermal expansion coefficient,$p^*$ is the pressure ,$\tau^*$ is the viscous stress term, $g^*$ is the gravitational vector,$c_p^*$ is the specific heat, $T^*$ is the temperature, $q^*$ is the heat flux vector. 

\bigskip

for Newtonian fluid, $$\tau^* = \mu^*(\nabla u^* +(\nabla u^*)^T)) - \frac{2}{3}\mu^*\nabla \cdot u^*I$$

By fourier law's, 
$$q^* =  k^* \nabla T^*$$

\bigskip

$\mu^*$ is the dynamic viscocity. $k^*$ is the thermal conductivity.

\bigskip

 The following are the quantities in equation of state, $\beta^* = -\frac{1}{\rho^*}\frac{\partial \rho^*}{\partial T^*}$ and $\alpha^* = \frac{1}{\rho^*}\frac{\partial \rho^*}{\partial p^*}$.

\subsection{Low Mach Number Asymtotic Analysis}

The work below uses low mach number symptotic analysis developed by Muller\cite{Muller}. We nondimensionalize the Equations by using reference quantities denoted by the subscript $\infty$,
e.g. farfield or stagnation conditions, and a typical length scale $L^*$ of the considered
flow. The thermodynamic reference quantities are assumed to be related by the equation of
state for a perfect gas. We have used perfect gas law because we are dealing with gases in combustion environment. We define the nondimensional quantities by:\\

\bigskip

$\rho = \frac{\rho^*}{\rho_\infty} $, $p = \frac{p^*}{p_\infty} $, $u = \frac{u^*}{u_\infty} $, $T = \frac{T^*}{T_\infty} $, $\mu = \frac{\mu^*}{\mu_\infty} $, $k = \frac{k^*}{k_\infty} $,$x = \frac{x^*}{L^*} $, $t = \frac{t^*}{L^*/u^*_{\infty}} $, $\beta^* = \frac{\beta}{\beta_{\infty}} $, 
\bigskip

$C_p^* = \frac{C_p}{{C_p}^*_{\infty}}$.

Using the relations above; we may write the nondimensional
Navier-Stokes equations and other equations of interest as follows:
\begin{eqnarray}
\frac{D \rho}{D t}\hspace{1mm} &=& -\hspace{1mm}\rho \hspace{1mm}\nabla \cdot u \\
\rho \frac{D u}{D t} \hspace{1mm} &=& -\frac{1}{M^2}\hspace{1mm}\nabla p\hspace{1mm}+\hspace{1mm}\frac{1}{Re}\nabla \cdot \tau + \frac{1}{Fr^2}\rho g \\
\rho C_p \frac{D T}{D t} \hspace{1mm}&=&\hspace{1mm} \frac{M^2}{Re \lambda}\hspace{1mm}\tau \nabla \cdot u \hspace{1mm}-\hspace{1mm}\frac{1}{Re Pr} \hspace{1mm}\nabla \cdot (k \nabla T) + \hspace{1mm} \frac{\beta T}{\lambda} \frac{D p}{D t}\hspace{1mm} \\
\frac{D \rho}{D t} &=& \hspace{1mm} - \beta T \frac{D \rho}{D t} + \alpha \rho \hspace{1mm}\frac{D p}{D t}
\end{eqnarray}
Where the following nondimensional quantities are

$$M= \frac{u_\infty}{a_\infty}$$ 
$$Re= \frac{u_\infty \rho_\infty L}{\mu_\infty}$$ 
$$Pr= \frac{{C_p}_\infty \mu_\infty L}{k_\infty}$$
$$Fr= \sqrt{\frac{u_\infty^2} {g_\infty L}}$$ 

 $a_{\infty}$ is the reference speed of sound, $M$ is the mach number, $Re$ is the reynolds number, $Pr$ is the prandtl number, $Fr$ is the froude number and $\lambda$ is defined as $\frac{{C_p}_\infty T_\infty }{a^2_\infty}$. 

 From ideal gas law, 

$$\beta^*= -\frac{1}{\rho^*}\frac{\partial \rho^*}{\partial T^*}$$ We non dimensional this term as follows 

$$\beta^*= -\frac{1}{\rho_\infty \rho}\frac{\rho_\infty}{T_\infty}\frac{\partial \rho}{\partial T}$$

$$\beta^*= \frac{1}{T_\infty } \beta$$ Where $\beta_\infty = \frac{1}{T_\infty}$
\bigskip

 From ideal gas law, 

$${a^*}^2= \frac{\partial p^*}{\partial \rho^*}$$ We non dimensional this term as follows 

$${a^*}^2= \frac{p_\infty}{\rho_\infty}\frac{\partial p}{\partial \rho}$$

$${a^*}^2= \frac{p_\infty}{\rho_\infty}a^2$$ Where $a_\infty = \sqrt{\frac{p_\infty}{\rho_\infty}}$.

\subsection{Asymptotic Analysis}
We do asymptotic analysis with respect to Mach number.
We use the following equation in terms of Mach number

$$\xi(x,t,M)=\xi^0(x,t)+M \xi^1(x,t)+M^2 \xi^2(x,t)+O(M^3)$$ \\

 Where $\xi$ can be $u,p,\rho$. We use non dimensional form of navier stokes equation do the asymptotic equation. 
We show the example of asymptotic analysis for mass conservation. 

$$\frac{D}{D t}(\rho^0+M \rho^1+M^2 \rho^2+O(M^3))\hspace{1mm}+\hspace{1mm}(\rho^0+M \rho^1+M^2 \rho^2+O(M^3)) \hspace{1mm}\nabla \cdot (u^0+M u^1+M^2 u^2+O(M^3)) = 0 $$

Collecting equal order terms together we get,

$$\left(\frac{D \rho^0}{D t}\hspace{1mm}+\hspace{1mm}\rho^0 \hspace{1mm}\nabla \cdot u^0\right) + M \left(\frac{\partial \rho^1 }{\partial t}+ +\hspace{1mm}\rho^1 \hspace{1mm}\nabla \cdot u^0++\hspace{1mm}\rho^0 \hspace{1mm}\nabla \cdot u^1\right) + O(M^2) = 0 $$ Therefore zeroth order terms can be written.
\begin{eqnarray}
\frac{D \rho^0}{D t}\hspace{1mm}+\hspace{1mm}\rho^0 \hspace{1mm}\nabla \cdot u^0 &=& 0
\end{eqnarray}

 The asymptotic relation for momentum are obtained are as follows.  The zero, first and second order momentum equation are written as follows respectively
\begin{eqnarray}
M^{-2}\nabla p^{(0)} &=& 0\\
M^{-1}\nabla p^{(1)} &=& 0\\
\rho^0 \frac{D u^0}{D t} \hspace{1mm} &=& -\hspace{1mm}\nabla p^2\hspace{1mm}+\hspace{1mm}\nabla \cdot \tau^0 + \rho^0 g
\end{eqnarray}

 The asymptotic relation for momentum are obtained are as follows.
\begin{eqnarray}
\rho^0 C_p^0\frac{D T^0}{D t} \hspace{1mm}&=&\hspace{1mm}-\hspace{1mm}\frac{1}{Re Pr} \hspace{1mm}\nabla.(k^0 \nabla T^0) + \hspace{1mm} \frac{\beta^0 T^0}{\lambda} \frac{d p^0}{d t}\hspace{1mm}
\end{eqnarray}
We have seen from momentum asymptotic analysis that $p^0=p^0(t)$ and therefore $$\frac{D p^0}{D t}= \frac{d p^0}{d t}$$. 
 
 The nondimensional equation of state is same as that of the dimensional form. Now same asymptotic expansion of the state equation is done in order to get zeroth order state equation which is given below. 
\begin{eqnarray}
\frac{D \rho^0}{D t} &=& \hspace{1mm} - \beta^0 T^0 \frac{D \rho^0}{D t} + \alpha^0 \rho^0 \hspace{1mm}\frac{D p^0}{D t}
\end{eqnarray}

 From mass conservation equation and since $p^0=p^0(t)$, we get the following equation
\begin{eqnarray}
- \nabla \cdot u^0 &=& \hspace{1mm} - \beta^0  \frac{D T^0}{D t} + \alpha^0  \hspace{1mm}\frac{d p^0}{d t}
\end{eqnarray}

 From energy conservation equation
\begin{eqnarray}
- \rho^0 C_p^0 \nabla \cdot u^0 &=& \hspace{1mm}  \frac{1}{Re Pr}\nabla \cdot q^0 + \left(\frac{C_p^0 \lambda -1 }{\lambda}  \right)\hspace{1mm}\frac{d p^0}{d t}
\end{eqnarray}

The final equation obtained by applying various thermodynamic properties are following
\begin{eqnarray}
-\nabla \cdot u &=& \frac{1}{p^0} \frac{d p^0}{d t} - \frac{1}{T}\frac{D T}{D t} \hspace{1mm}\\
\rho \frac{D u}{D t} \hspace{1mm} &=& -\frac{1}{M^2}\hspace{1mm}\nabla p\hspace{1mm}+\hspace{1mm}\frac{1}{Re}\nabla \cdot \tau + \frac{1}{Fr^2}\rho g \\
\rho C_p \frac{D T}{D t} \hspace{1mm}&=&\hspace{1mm} -  \frac{1}{Re Pr}\nabla\cdot (k \nabla T) + \hspace{1mm} \frac{1}{\gamma -1} \frac{d p^0}{d t}\hspace{1mm}\\
\end{eqnarray}

